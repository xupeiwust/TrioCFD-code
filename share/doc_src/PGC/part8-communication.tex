\rhead{COMMUNICATION}

\lettrine[lines=2,slope=0pt,nindent=4pt]{\textbf{P}}{lusieurs} canaux de communications ont été mis en place progressivement autour de TrioCFD
afin de permettre aux utilisateurs et développeurs de TrioCFD d'échanger sur le code, sur les difficultés qu'ils rencontrent, d'exprimer
leurs besoins autant en terme de développements que d'outils, mais également de les informer sur les capacités et évolutions de
TrioCFD. Chaque canal est dédié à un type d'information particulier avec une description de chacun d'eux ci-dessous.

\chapter{\label{chapitre:site}Site TrioCFD}
\lhead{Site TrioCFD}
\rhead{COMMUNICATION}

Le site TrioCFD a été construit en 2019 et est accessible à l'adresse suivante :
"\url{http://triocfd.cea.fr/Pages/Presentation/TrioCFD_code.aspx}". Il est actuellement construit comme illustré sur
la figure \ref{figure:structure_site_trio}. Les rubriques encadrées en rouge ne sont actuellement pas renseignées
et celles en jaune, sont en cours de rédaction. Les rubriques non encadrées sont achevées.\newline

\begin{center}\includegraphics[width=14cm]{pictures/site_Trio.png}\end{center}
\begin{center}\captionof{figure}{\label{figure:structure_site_trio}Structure actuelle du site TrioCFD.}\end{center}

L'objectif du site est de présenter de fa\c con générale le code TrioCFD, les modèles qui y sont implémentés et les applicatifs pour lequel il est utilisé/applicable. Un bon nombre d'articles ou de rapport de thèses présentant les travaux effectués avec TrioCFD y sont disponibles ainsi que la documentation du code (rapport de validation, documentation des modèles, manuel de référence). La page d'accueil regroupe les actualités du code comme l'annonce de séminaire, de sortie de version,... Le lien vers GitHub sur lequel le code source de TrioCFD est disponible est également référencé.\newline
Le site s'adresse autant aux utilisateurs interne CEA du code qu'aux utilisateurs universitaires ou industriels. Un formulaire de contact est à la disposition des utilisateurs (pour ceux ne connaissant pas l'adresse du projet) afin qu'ils soient nénamoins en mesure de prendre contact avec l'équipe en cas de question ou de problème.

Pour l'instant, le site est très majoritairement en fran\c cais et toutes les parties nommées ci-dessus ne sont pas achevées. Un travail de refonte du site est actuellement en cours afin notamment de l'enrichir avec les nouveaux applicatifs pour lesquels TrioCFD est utilisé. Lorsque le nouveau format du site sera terminé, celui-ci sera
exclusivement en anglais. Une première maquette de la nouvelle structure du site est donnée en figure \ref{figure:nouvelle_structure_site_trio}

\begin{center}
\includegraphics[width=10cm]{pictures/nouveau_site_trio.png}
\end{center}
\begin{center}
\captionof{figure}{\label{figure:nouvelle_structure_site_trio}$1^{\text{ère}}$ maquette de la nouvelle structure du site TrioCFD.}
\end{center}

Il s'agit là d'une première ébauche qui pourra être quelque peu modifiée lors de son implémentation. Cette nouvelle structure a pour but de créer une cartographie complète et structurée des domaines d'utilisation de TrioCFD, des modèles utilisés pour chaque applicatif et des illustrations de ceux-ci par des exemples. Dans sa forme finale, le site présentera également l'équipe TrioCFD, le formulaire de contact sera conservé ainsi que la base de données (\texttt{DATABASE}) et les publications relatives à TrioCFD.

\chapter{GT et réunion de développement}
\lhead{GT et réunion de développement}
\rhead{COMMUNICATION}

Les réunions régulière concernant TrioCFD sont résumées dans le tableau \ref{tab:reunions}
\begin{landscape}

\begin{table}[p]
\centering
\caption{}
\label{tab:reunions}
\begin{tabularx}{\textheight}{|>{\hsize=1.0\hsize\linewidth=\hsize}X|>{\hsize=2.0\hsize\linewidth=\hsize}X|>{\hsize=0.5\hsize\linewidth=\hsize}X|>{\hsize=0.5\hsize\linewidth=\hsize}X|>{\hsize=1.0\hsize\linewidth=\hsize}X|}
\hline 
Réunion & Périmètre & Fréquence & Durée & Participants \\

\hline 

\centering
Réunion technique TrioCFD  & 
\begin{itemize}
\item Suivi de version, bilan des intégrations et développements en cours
\item Bonnes pratiques de développement
\item Zoom dev (défini en amont, 2-3 slides informels)
\end{itemize}
&
1/mois
& 
10 m

30 m

20 m
& 
Développeurs TrioCFD (Permanents LDEL et  externes et doctorants si concernés par l'ordre ju jour) \\

\hline 

\centering
Réunion stratégique codes LDEL & Stratégie des codes, partenariats, sollicitations, couplage & ~2/an & 2 h & Permanents LDEL sur TrioCFD \\

\hline 

\centering
Réunions thématiques & 

4 thématiques physiques en roulement :
\begin{itemize}
\item Turbulence (monophasique) + IFS
\item Front Tracking 
\item R\&D numérique
\item CMFD -- Neptune - EOS
\end{itemize}
& 
En roulement 
2/mois
&
1 h jeudi matin
sans PC
& 
STMF
permanents et non-permanents \\
\hline 
\end{tabularx}
\end{table}
\end{landscape}
Si un besoin particulier est exprimé, une réunion supplémentaire peut être provoquée, ou la fréquence augmentée sur une période donnée.

\chapter{Séminaires}
\lhead{Séminaires}
\rhead{COMMUNICATION}

Différents formats de séminaires rythment l'année :
\begin{itemize}[label=$\Rightarrow$, font=\LARGE]
\item \textbf{Séminaire des étudiants :} Au départ des stagiaires, doctorants ou post-doctorants, un séminaire est organisé en interne CEA afin qu'ils présentent les travaux qui ont été effectués dans ce cadre. En ce qui concerne les stagiaires, le séminaire regroupe plusieurs présentations traitant d'une même thématique.
Ainsi, plusieurs séminaires de ce type seront organisés chaque année. Pour les doctorants, le séminaire dédié lui permet un entra\^inement en grandeur nature pour sa soutenance de thèse.
\item \textbf{Séminaire des permanents :} Lorsque un sujet de recherche, une étude ou un livrable est arrivé à maturité, un séminaire spécifique est organisé en interne CEA afin d'échanger sur le sujet.
\item \textbf{Séminaire TrioCFD :} Tous les 2 ans, le séminaire TrioCFD est organisé avec l'ensemble des utilisateurs internes ou externes CEA. La journée s'articule autour de présentations techniques autant sur des applications pour lesquelles TrioCFD est utilisé pour la modélisation que sur des développements majeurs de nouvelles fonctionnalités. C'est une occasion pour les utilisateurs de découvrir les avancées du code sur les deux dernières années et d'avoir un pannel complet des applicatifs. Une partie importante de la journée est également dédiée aux échanges et au partage d'expériences.
\end{itemize}


