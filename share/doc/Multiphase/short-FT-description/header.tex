%% The amssymb package provides various useful mathematical symbols
% \usepackage{amssymb}
% %% The amsthm package provides extended theorem environments
% %% \usepackage{amsthm}
% 
% %% The lineno packages adds line numbers. Start line numbering with
% %% \begin{linenumbers}, end it with \end{linenumbers}. Or switch it on
% %% for the whole article with \linenumbers.
% %% \usepackage{lineno}
% 
\usepackage[utf8x]{inputenc}
\usepackage{changes}
% \usepackage{rotating} % Pour tourner le tableau
% % \usepackage{multirow} % Pour le multi-ligne
\usepackage{csquotes} % Pour \guillemot
\usepackage{booktabs} % \toprule, \midrule
\usepackage{amsmath}  % \text{...}
\usepackage{multirow} % \multirow
% \usepackage{pdflscape} % \begin{landscape}
\usepackage{ulem}      % Pour le \uwave de revoir... 
\usepackage{subfig}    % \subfloat
% \showcaptionsetup[uniq]{subfloat}
\captionsetup[subfloat]{font={rm,md,sc},
captionskip=0pt, farskip=0pt, nearskip=0pt,
margin=0pt, parskip=0pt,
hangindent=0pt, indention=0pt, singlelinecheck=true}%,aboveskip=0pt,parskip=0pt}
% 
\usepackage{xspace}
\usepackage{calrsfs}
\usepackage{graphicx}
\usepackage{tikz}
% \usepackage{pgf,tikz,pgfplots}
% \pgfplotsset{compat=1.15}
% \usepackage{mathrsfs}
% \usetikzlibrary{arrows}

% \usetikzlibrary{arrows,patterns,plotmarks,shapes,snakes,er,3d,automata,backgrounds,topaths,trees,petri,mindmap}
% % % % \usetikzlibrary{arrows,patterns,plotmarks,shapes,decorations,er,3d,automata,backgrounds,topaths,trees,petri,mindmap}
% % % % \def\smallbluebox{
% % % %    \begin{tikzpicture}[scale=.2]
% % % %      \draw[very thick,fill=blue!10!white,draw=blue!50] (0,0) rectangle
% % % % (1,1) {};
% % % %    \end{tikzpicture}}

\newif\ifrevisionmode
\revisionmodetrue % Version Revision 
% \revisionmodefalse % Version Publique
\ifrevisionmode
 \newcommand{\ajout}[1]{{\color{blue}\uuline{#1}}}
 \newcommand{\correct}[1]{{\color{red}\sout{#1}}}
 \newcommand{\revoir}[1]{{\color{green}\uwave{#1}}}
 \def\code#1{{\color{blue}{\texttt{\detokenize{#1 }}}}}
%  \newcommand{\code}[1]{{\color{blue}{Code information: }{#1}}}
\else
 \newcommand{\ajout}[1]{{#1}}
 \newcommand{\correct}[1]{{}}
 \newcommand{\revoir}[1]{{}}
 \newcommand{\code}[1]{{}}
\fi

\newcommand{\tbd}[0]{\revoir{To be written elsewhere? ICMF, journal\dots}}

% Shortcuts : 
\renewcommand{\emph}[1]{\textit{#1}}
\newcommand{\ncfd}[0]{NEPTUNE\_CFD\xspace}
\def\lrp#1{\left( #1\right)}
\def\pard#1#2{\frac{\partial #1}{\partial #2}}
\def\lrc#1{\left\langle #1\right\rangle}
\def\lrb#1{\left[ #1\right]}
% % \def\tens#1{\uuline{#1}}
% \def\tens#1{\underline{#1}}
\newcommand{\bfs}[1]{\mathbf{#1}}
\renewcommand{\vec}[1]{\bfs{#1}}
\newcommand{\ve}[1]{\vec{#1}}
\newcommand{\into}[0]{\hookleftarrow}
\newcommand{\Rey}[0]{\text{Re}}
\newcommand{\We}[0]{\text{We}}
\newcommand{\Eo}[0]{\text{Eo}}
\newcommand{\Fr}[0]{\text{Fr}}
% \newcommand{\Pe}[0]{\text{Pe}}
% \newcommand{\Ja}[0]{\text{Ja}}
\newcommand{\Fo}[0]{\text{Fo}}
% \newcommand{\Nu}[0]{\text{Nu}}
\newcommand{\At}[0]{\text{At}}
\newcommand{\Ar}[0]{\text{Ar}}
\newcommand{\up}[0]{\text{w.u.}}
\newcommand{\ua}[0]{\text{d.u.}}
\newcommand{\Mo}[0]{\text{Mo}}
\newcommand{\Ca}[0]{\text{Ca}}
\newcommand{\Eotvos}[0]{E\"{o}tv\"{o}s\xspace}
% % 
\def\red#1{\textcolor{red}{ #1}}
\def\f#1{\overline{ #1}}
\def\fii#1{\overline{ #1}^i}
%
% \def\ff#1{\overline{\overline{#1}}}
\def\fk#1{\overline{ #1}^k}
\def\fl#1{\overline{ #1}^l}
\def\fv#1{\overline{ #1}^v}
% %
\def\fuk{\fk{\vec{u}}_k}
\def\fpl{\fl{p}_l}
\def\fpv{\fv{p}_v}
\def\fpk{\fk{p}_k}
\def\ful{\fl{\vec{u}}_l}
\def\fuv{\fv{\vec{u}}_v}
% \def\fu#1{\fk{\vec{u}}_{#1}}
% \def\fupk{\fk{\vec{u}'}_k}
% \def\frk{\fk{\vec{u}_k'\vec{u}_k'}}
% \def\psat{p^\text{sat}}
% % 
\def\norm#1{\parallel \! #1 \! \parallel} % Norme
% \def\Id{\vec{\text{Id}}}
\def\Cg{\lrc{\vec{\mathcal{C} }_g}}  %\def\Cg{\tilde{\vec{\mathcal{C} }}}
\def\xr{\vec{x}_r}
\newcommand{\tr}[1]{\text{tr}\lrp{#1}}
% % 
\newcommand{\tento}[1]{\, 10^{#1}}
\newcommand{\qetq}[0]{{\quad\mbox{and}\quad}}
\newcommand{\qouq}[0]{{\quad\mbox{or}\quad}}
\newcommand{\qq}[1]{\quad\mbox{#1}\quad}
\newcommand{\ie}[0]{\textit{i.\,e.,\,}}
\newcommand{\eg}[0]{\textit{e.\,g.,\,}}
% \def\eref#1{Eq.~\eqref{#1}}
\def\erefs#1#2{\eqref{#1} et~\eqref{#2}}
% \def\erefss#1#2#3{\eqref{#1}, \eqref{#2} et~\eqref{#3}}
\def\eqref#1{(\ref{#1})}
\def\refs#1#2{\eqref{#1} and~\eqref{#2}}

\newcommand{\guillemot}[1]{\blockquote{#1}}
% \newcommand{\guillemot}[1]{\textquote{#1}}
% \newcommand{\citet}[1]{\citeauthor{#1}~\cite{#1}}
\renewcommand{\otimes}{\,}

\newcommand{\Cp}[0]{c_p}
\newcommand{\rhoCp}[0]{\rho\,c_p}
\newcommand{\Cpl}[0]{c_{pl}}
\newcommand{\Cpv}[0]{c_{pv}}
\newcommand{\Cpk}[0]{c_{pk}}
\newcommand{\Ti}[0]{{T^{i}}}
\newcommand{\Tsat}[0]{{T^{\mathrm{sat}}}}
\newcommand{\Tle}[0]{{T_l^{ext}}}
\newcommand{\Lvap}[0]{\mathcal{L}^{vap}}

% Creation de double brackets :
\newsavebox{\sembox}
\newlength{\semwidth}
\newlength{\boxwidth}
\newcommand{\jump}[1]{%
\sbox{\sembox}{\ensuremath{#1}}%
\settowidth{\semwidth}{\usebox{\sembox}}%
\sbox{\sembox}{\ensuremath{\left[\usebox{\sembox}\right]}}%
\settowidth{\boxwidth}{\usebox{\sembox}}%
\addtolength{\boxwidth}{-\semwidth}%
\left[\hspace{-0.3\boxwidth}%
\usebox{\sembox}%
\hspace{-0.3\boxwidth}\right]%
}
\newcommand{\svm}[0]{\jump{1/\rho}} 
\newcommand{\nv}[0]{\vec{n}^\textrm{v}} 
\newcommand{\ns}[0]{\vec{n}^s} 
\newcommand{\q}[0]{{\dot{q}}}         % flux de chaleur
\newcommand{\Q}[0]{Q}                 % flux de chaleur lineique
\newcommand{\qlog}[0]{{\mathring q}}  

\usepackage{geometry}
 \geometry{
 a4paper,
 total={170mm,257mm},
 left=20mm,
 top=20mm,
 }
 
 
 % la bibliographie
 \usepackage[%
 natbib=false,        % If you want to keep on using the natbib names even with
 %                     biblatex, you can load biblatex with the option
 %                     natbib=true to load the natbib compatibility mode. Note
 %                     that the natbib compatibility mode will also change
 %                     nameyeardelim, see Is there a disadvantage to using
 %                     natbib=true with biblatex?.
 %                     https://tex.stackexchange.com/q/149313/35864
 backend=biber,      % On utilise biber pour la compilation biblatex
 citestyle=numeric,  % citation de la forme [1] ou pour plusieurs, [1], [2]
 bibstyle=numeric,   % complementaire à citestyle numeric pour la biblio finale
 sorting=anyt,        % Tri dans l'ordre : nom, année, titre.
 maxnames=2,         % Les citations dans le texte avec \textcite n'affiche que
 %                     le nom du premier auteur et met un "et al.".
 maxbibnames=99,     % Limite haute pour mettre tous les noms dans la biblio (ne met pas de et al. dans la bibliographie)
 giveninits=true,    % ne mets que les initiales
 language=french,    % test du français
 alldates=year,      % la date n'affiche que l'année
 isbn=false,         % Ne pas afficher l'isbn
 url=false,          % Ne pas afficher l'url
 doi=true,          % Ne pas afficher le lien DOI
 eprint=false,       % Ne pas afficher le numéro eprint
 %refsegment=chapter, % pour faire des bibliographies par chapitre
 backref=true       % Ajouter une mention renvoyant où la publication a été citée
 ]{biblatex}
 \renewbibmacro{in:}{}
 
 
 \newbibmacro{stringanddoiurlisbn}[1]{%
 	\iffieldundef{doi}{%
 		\iffieldundef{url}{%
 			\iffieldundef{isbn}{%
 				\iffieldundef{issn}{%
 					#1%
 				}{\href{https://books.google.com/books?vid=ISSN\thefield{issn}}{#1}}%
 			}{\href{https://books.google.com/books?vid=ISBN\thefield{isbn}}{#1}}%
 		}{\href{\thefield{url}}{#1}}%
 	}{\href{https://doi.org/\thefield{doi}}{#1}}%
 }
 
 \DeclareFieldFormat{title}{\usebibmacro{stringanddoiurlisbn}{\mkbibemph{#1}}}
 \DeclareFieldFormat[article,incollection,inproceedings,thesis,inbook]{title}%
 {\usebibmacro{stringanddoiurlisbn}{\mkbibquote{#1}}}
 